%\documentclass{article}
\documentclass[12pt]{article}
\usepackage{latexsym}
\usepackage{amsmath}
\usepackage{amssymb}
\usepackage{relsize}
\usepackage{geometry}
\geometry{letterpaper}

%\usepackage{showlabels}

\textwidth = 6.0 in
\textheight = 8.5 in
\oddsidemargin = 0.0 in
\evensidemargin = 0.0 in
\topmargin = 0.2 in
\headheight = 0.0 in
\headsep = 0.0 in
%\parskip = 0.05in
\parindent = 0.35in


%% common definitions
\def\stackunder#1#2{\mathrel{\mathop{#2}\limits_{#1}}}
\def\beqn{\begin{eqnarray}}
\def\eeqn{\end{eqnarray}}
\def\nn{\nonumber}
\def\baselinestretch{1.1}
\def\beq{\begin{equation}}
\def\eeq{\end{equation}}
\def\ba{\beq\new\begin{array}{c}}
\def\ea{\end{array}\eeq}
\def\be{\ba}
\def\ee{\ea}
\def\stackreb#1#2{\mathrel{\mathop{#2}\limits_{#1}}}
\def\Tr{{\rm Tr}}
\newcommand{\gsim}{\lower.7ex\hbox{$
\;\stackrel{\textstyle>}{\sim}\;$}}
\newcommand{\lsim}{\lower.7ex\hbox{$
\;\stackrel{\textstyle<}{\sim}\;$}}

%%
\newcommand{\p}{\partial}
\newcommand{\wt}{\widetilde}
\newcommand{\ov}{\overline}
\newcommand{\mc}[1]{\mathcal{#1}}
\newcommand{\md}{\mathcal{D}}
\newcommand{\ph}{\phantom}

\newcommand{\GeV}{{\rm GeV}}
\newcommand{\eV}{{\rm eV}}
\newcommand{\Heff}{{\mathcal{H}_{\rm eff}}}
\newcommand{\Leff}{{\mathcal{L}_{\rm eff}}}
\newcommand{\el}{{\rm EM}}
\newcommand{\uflavor}{\mathbf{1}_{\rm flavor}}
\newcommand{\lgr}{\left\lgroup}
\newcommand{\rgr}{\right\rgroup}

\newcommand{\Mpl}{M_{\rm Pl}}
\newcommand{\suc}{{{\rm SU}_{\rm C}(3)}}
\newcommand{\sul}{{{\rm SU}_{\rm L}(2)}}
\newcommand{\sutw}{{\rm SU}(2)}
\newcommand{\suth}{{\rm SU}(3)}
\newcommand{\ue}{{\rm U}(1)}
%%%%%%%%%%%%%%%%%%%%%%%%%%%%%%%%%%%%%%%
%  Slash character...
\def\slashed#1{\setbox0=\hbox{$#1$}             % set a box for #1
   \dimen0=\wd0                                 % and get its size
   \setbox1=\hbox{/} \dimen1=\wd1               % get size of /
   \ifdim\dimen0>\dimen1                        % #1 is bigger
      \rlap{\hbox to \dimen0{\hfil/\hfil}}      % so center / in box
      #1                                        % and print #1
   \else                                        % / is bigger
      \rlap{\hbox to \dimen1{\hfil$#1$\hfil}}   % so center #1
      /                                         % and print /
   \fi}                                        %

%%EXAMPLE:  $\slashed{E}$ or $\slashed{E}_{t}$

%%

\newcommand{\LN}{\Lambda_\text{SU($N$)}}
\newcommand{\sunu}{{\rm SU($N$) $\times$ U(1)} }
\newcommand{\sunun}{{\rm SU($N$) $\times$ U(1)}}
\def\cfl {$\text{SU($N$)}_{\rm C+F}$ }
\newcommand{\mUp}{m_{\rm U(1)}^{+}}
\newcommand{\mUm}{m_{\rm U(1)}^{-}}
\newcommand{\mNp}{m_\text{SU($N$)}^{+}}
\newcommand{\mNm}{m_\text{SU($N$)}^{-}}
\newcommand{\AU}{\mc{A}^{\rm U(1)}}
\newcommand{\AN}{\mc{A}^\text{SU($N$)}}
\newcommand{\aU}{a^{\rm U(1)}}
\newcommand{\aN}{a^\text{SU($N$)}}
\newcommand{\baU}{\ov{a}{}^{\rm U(1)}}
\newcommand{\baN}{\ov{a}{}^\text{SU($N$)}}
\newcommand{\lU}{\lambda^{\rm U(1)}}
\newcommand{\lN}{\lambda^\text{SU($N$)}}
%\newcommand{\Tr}{{\rm Tr\,}}
\newcommand{\bxir}{\ov{\xi}{}_R}
\newcommand{\bxil}{\ov{\xi}{}_L}
\newcommand{\xir}{\xi_R}
\newcommand{\xil}{\xi_L}
\newcommand{\bzl}{\ov{\zeta}{}_L}
\newcommand{\bzr}{\ov{\zeta}{}_R}
\newcommand{\zr}{\zeta_R}
\newcommand{\zl}{\zeta_L}
\newcommand{\nbar}{\ov{n}}

\newcommand{\loU}{\lambda_0^{\rm U(1)}}
\newcommand{\llU}{\lambda_1^{\rm U(1)}}
\newcommand{\loN}{\lambda_0^\text{SU($N$)}}
\newcommand{\llN}{\lambda_1^\text{SU($N$)}}
\newcommand{\poU}{\psi_0^{\rm U(1)}}
\newcommand{\plU}{\psi_1^{\rm U(1)}}
\newcommand{\poN}{\psi_0^\text{SU($N$)}}
\newcommand{\plN}{\psi_1^\text{SU($N$)}}

\newcommand{\CPC}{CP($N-1$)$\times$C }
\newcommand{\CPCn}{CP($N-1$)$\times$C}

\newcommand{\MN}{M_\text{SU($N$)}}
\newcommand{\MU}{M_{\rm U(1)}}

\newcommand{\tgamma}{\wt{\gamma}}
\newcommand{\btgamma}{\ov{\tgamma}}

\begin{document}

\centerline{\Huge\bf The Central Idea}

\vspace{2.0mm}

\begin{minipage}[b]{0.76cm}
                \fontseries{bx}
                \fontfamily{pnc}
                \fontsize{44}{12}
                \selectfont
                T
\end{minipage}
\begin{minipage}[b]{3.36cm}
        	\Large\fontfamily{pnc}\selectfont
		\textsc{he kinetic}
\end{minipage}
	mixing matrix can be specified in an explicitly diagonalized form
\beq
\label{M}
	\mc{M}  ~~~~=~~~~  
			\lgr\; \begin{matrix}
				\cos\, \theta    &    - \sin\, \theta \\
				\sin\, \theta    &    \ph- \cos\, \theta
			\end{matrix} \;\rgr
			\lgr\; \begin{matrix}
			     		a^2    &        \\
					       &    b^2
			\end{matrix} \;\rgr
			\lgr\; \begin{matrix}
				\ph-\cos\, \theta    &    \sin\, \theta \\
				- \sin\, \theta    &    \cos\, \theta
			\end{matrix} \;\rgr
			.
\eeq

	The quantity $ a^2 $ will be the effective coupling constant of the photon, and therefore
	we choose 
\[
	a^2  ~~\simeq~~  1\,.
\]
	Meanwhile, the quantity $ b^2 $ will be the coupling constant of the hidden photon.
	We would like to have the hidden photon completely decoupled by the end of inflation, and also
	never to be at strong coupling.
	Therefore, we choose $ b^2 $ to be growing,
\[
	b^2  ~~\propto~~ a^{2n}(\tau)\,,
\]
	where $ a(\tau) $ is the scale factor\footnote{
	To distinguish the scale factor $ a(\tau) $ from the inverse coupling constant $ a^2 $ 
	we will always write it as $ a(\tau) $ --- a function of $ \tau $.
	Note that on the contrary, $ a^2 $ is practically constant in our scenario.}.
	One cannot consistently impose the alternative condition, 
$ b^2 ~~\propto~~ a^{-2n}(\tau) $.
	In this case $ b(\tau) $ would ultimately become very small, 
	and the quantum corretions would take over and alter its value.

	The kinetic matrix \eqref{M} can be evaluated to the following form,
\beq
\label{Mready}
	\mc{M}  ~~=~~  
			\lgr\;\;\, \begin{matrix}
				    1           \,&\,    \epsilon    \\
				    \epsilon    \,&\,    f^2
			     \end{matrix} \;\;\rgr,
\eeq
	where
\[
	1  ~~\ll~~  f^2
	\qquad\qquad
	\text{and}
	\qquad\qquad
	\epsilon  ~~\ll~~  f\,.
\]
	We define our Lagrangian such that this matrix mixes the observable photon with the hidden photon.
	The small value of the coupling of the hidden photon to matter is ensured by the growing 
$ f^2(\tau) ~\simeq~ b^2(\tau) $.
	On the other hand, the coupling constant of the photon to matter must remain 
	of order one at all times. 
%	For this reason, $ \theta $ cannot be a constant, but must decrease with time.
	Exactly how this is realized we will show in the next subsection.

\pagebreak

\vspace{1.0cm}
\centerline{*\qquad\qquad\qquad*\qquad\qquad\qquad*}
\vspace{1.0cm}

	We have already defined that 
\[
	a^2    ~~=~~    1\,,
	\qquad\qquad
	\text{and}
	\qquad\qquad
	b^2    ~~\simeq~~    a^{2n}(\tau).
\]
	Let us accept that these are in fact exact equalities.
	For the coupling constant of the real photon to remain of order one at all times, 
	$ \theta $ must decrease in time,
\[
	\theta    ~~\simeq~~    \sin\,\theta    ~~\propto~~    a^{-\, n \,-\, \lambda}(\tau) \,.
\]
	Here $ \lambda $ is a small ``excess'' power, say $ \lambda  ~\sim~ 1/2 $.
	These requirements ensure that the mixing matrix takes the form as that of Eq.~\eqref{Mready}.

	The time dependences of $ \epsilon(\tau) $ and $ f^2(\tau) $ are,
\[
	f^2    ~~=~~    a^{2n}(\tau)\,
\]
	and
\[
	\epsilon    ~~=~~    -\, a^{n \,-\, \lambda}(\tau)\,.
\]
	The condition $ \epsilon ~\ll~ f(\tau) $ is also met.

	At this point we can write down the Lagrangian which we have in mind,
\[
	\mc{L}    ~~=~~    \lgr F_{\mu\nu}  \quad G_{\mu\nu} \rgr  \mc{M}  
			\lgr \begin{matrix}
			     		F_{\mu\nu}  \\
					G_{\mu\nu}
			     \end{matrix} \rgr
			~~+~~
			F^\mu\, J_\mu\,,
\]
	where $ J^\mu $ is the matter current, and we have suppressed factors like $ 1/4\pi $.
	It is our free will to define how the gauge fields couple to matter.
	Explicitly, $ F^\mu $ becomes the real photon and $ G^\mu $ the hidden one. 
	The behaviour of $ F^\mu $ is slightly altered as compared to that of the usual U(1) photon by the presence of $ G^\mu $.

	
\vspace{1.0cm}
\centerline{*\qquad\qquad\qquad*\qquad\qquad\qquad*}
\vspace{1.0cm}

	Here we sketch the underlying ideas of the scenario.
	The kinetic term written explicitly is
\beq
	\lgr F_{\mu\nu}\, \quad G_{\mu\nu} \rgr 
	\lgr\; \begin{matrix}
				1          &    \epsilon  \\
				\epsilon   &    f^2
	\end{matrix} \;\rgr
	\lgr\, \begin{matrix}
		F_{\mu\nu}  \\
		G_{\mu\nu}
	     \end{matrix} \,\rgr,
\eeq
	where again $ F_{\mu} $ and $ G_{\mu} $ are the real and the hidden photons.
	Although we will not need to diagonalize the kinetic matrix in the actual calculations, 
	we assume we do so just to provide a supporting argument.

	The form of the kinetic matrix \eqref{M} already gives the diagonalized form,
\beq
\label{diag}
	a^2\, A_{\mu\nu}^2  ~~+~~  b^2\, B_{\mu\nu}^2  ~~+~~  \text{{\small one-derivative} \;and\; {\small zero-derivative terms}.}
\eeq
	Here $ A_{\mu\nu} $ and $ B_{\mu\nu} $ are the rotated fieldstrengths.
	Now the rotation $ \sin \theta $ is applied to the gauge fields $ A_\mu $ and $ B_\mu $ rather than
	to the fieldstrengths.
	Since this rotation is time-dependent, it produces terms where less than two derivatives 
	are applied to $ A_{\mu} $ and $ B_{\mu} $.
	Ignore these terms now.

	What we have come to, is two decoupled instances of electrodynamics, one with a finite gauge coupling,
	the other one with a gauge coupling becoming very small.
	It is known that when the coupling constant decreases with time ({\it i.e.} the inverse coupling $ b^2(\tau) $ increases),
	the electrodynamics {\it can} produce the magnetic field for itself.
	This way, $ B_{\mu} $ generates the magnetic field $ \vec{B}_{(B)} $, 
	and it does not suffer from the strong coupling problem ---
	there is no constraint that $ b^2(\tau) $ nowadays be of order one. 

	But then, as we said, $ A_\mu $ and $ B_\mu $ are not the observable fields --- 
	rather, $ F_\mu $ and $ G_\mu $ are.
	Since the observable fields are linear combinations of $ A_\mu $ and $ B_\mu $, 
	by generating $ \vec{B}_{(B)} $ we also generate $ \vec{B}_{(F)} $, 
	that is, the observable magnetic field.

	Although these arguments are heuristic, we do expect them to give the rough idea 
	of the dynamics of the fields and expect that the extra terms in Eq.~\eqref{diag} can be neglected in some sense.
	Indeed, if the two instances of electrodynamics were completely decoupled, 
	the two fields would develop by themselves. 
	The mixing between them is so weak that hardly it can affect their dynamics on a significant level.

	The actual calculations will not rely on this discussion.


\vspace{1.0cm}
\centerline{*\qquad\qquad\qquad*\qquad\qquad\qquad*}
\vspace{1.0cm}

	
	Let us call the undiagonalizeds fields $ \Phi_\mu $,
\beq
	\lgr  F_\mu\,, \quad G_\mu  \rgr    ~~\equiv~~    \Phi_\mu\,.
\eeq


	As we know, at the time when a mode has exited the horizon, the equation of motion for
	it can be written as
\beq
	\big(\, \mc{M} \cdot \Phi^\prime \,\big)^\prime    ~~=~~    0\,,
	\qquad\qquad
	\text{or}
	\qquad\qquad
	\mc{M} \cdot \Phi^\prime    ~~=~~    \text{const}.
\eeq
	Here we assume that $ \Phi $ is written in terms of polarizations $ \Phi_{\pm} $.

	Fixing some time $ \tau_* $, we have
\beq
	\mc{M} \cdot \Phi^\prime    ~~=~~    \mc{M}_* \cdot \Phi_*^\prime
	\qquad\qquad
	\text{for that $\tau_*$}.
\eeq

	This leads to the differential equation
\beq
\label{diffeq}
	\Phi^\prime  ~~=~~ \mc{M}^{-1} \cdot \mc{M}_*\, \Phi_*^\prime\,.
\eeq

	The right hand side of this equation is in principle a known function of time, as long as
	we find the inverse of matrix $ \mc{M} $.
	The rest of the right hand side is just a set of constants.
	We expect the inverse of $ \mc{M} $ to be expressible as a combination of powers
	of $ a(\tau) $ --- that is, as a combination of powers of $ \tau $ itself.


\vspace{1.0cm}
\centerline{*\qquad\qquad\qquad*\qquad\qquad\qquad*}
\vspace{1.0cm}

	
	Quite generally,
\beq
\label{invM}
	\mc{M}^{-1}   ~~=~~    \frac{1}{f^2 \,-\, \epsilon^2}\,
					\lgr \begin{matrix}
						\ph{-} f^2    &    -\epsilon    \\
						-\epsilon   &    \ph- 1
					\end{matrix} \;\; \rgr  .
\eeq
	
	The denominator can just be expanded,
\beq
	\frac{1}{f^2 \,-\, \epsilon^2}    ~~=~~    
		\frac{1}{f^2}\, 
		\lgr  1  ~~+~~  \frac{\epsilon^2}{f^2}  ~~+~~  \frac{\epsilon^4}{f^4}  ~~+~~  \dots \rgr.
\eeq
	This gives the inverse matrix \eqref{invM} the form of a series in powers of the conformal time.
	
	Using this, we re-write the system of equations \eqref{diffeq} in terms of the conformal time.
	The leading terms in the system generally depend on the relative magnitudes of $ n $ and $ \lambda $.
	Assuming, for example $ \lambda \ll n $, the system, written in terms of polarizations, is
\begin{align}
%
\notag
	F_\lambda^\prime    & ~~=~~    \text{\small const}  ~~+~~  \text{\small const} \cdot \tau^{2 \lambda}  ~~+~~  \dots
	\\[3.4mm]
%
	G_\lambda^\prime    & ~~=~~    \text{\small const} \cdot \tau^{n \,+\, \lambda}  ~~+~~  \text{\small const} \cdot \tau^{n \,+\, 3\lambda}
					~~+~~  \dots\,.
\end{align}

%\begin{align}
%%
%\notag
%	F_\lambda^\prime    & ~~=~~    \ph{\frac{\epsilon}{f^2} \cdot} \text{\small const}  ~~+~~  \frac{\epsilon}{f^2} \cdot \text{\small const}  
%	\\
%%
%	G_\lambda^\prime    & ~~=~~    \frac{\epsilon}{f^2} \cdot \text{\small const}  ~~+~~ \frac{1}{f^2} \cdot \text{\small const}\,.
%\end{align}
%	
%
%	Now, taking into account that $ a(\tau) \,\propto\, 1/\tau $, let us re-write the equations in $ \tau $,
%\begin{align}
%%
%\notag
%	F_\lambda^\prime    & ~~=~~    \text{\small const}  ~~+~~  \text{\small const} \cdot \tau^{n \,-\, \lambda} 
%	\\[3.4mm]
%%
%	G_\lambda^\prime    & ~~=~~    \text{\small const} \cdot \tau^{n \,-\, \lambda}  ~~+~~  \text{\small const} \cdot \tau^{2n}\,.
%\end{align}

	Integration of these equations gives powers of $ \tau $ one greater,
\begin{align}
%
\notag
	F_\lambda    & ~~=~~    \text{\small const}  ~~+~~  \text{\small const} \cdot \tau  ~~+~~  \text{\small const} \cdot \tau^{2 \lambda \,+\, 1} 
	\\[3.8mm]
%
\label{syssol}
	G_\lambda    & ~~=~~    \text{\small const}  ~~+~~  
				\text{\small const} \cdot \tau^{n \,+\, \lambda \,+\, 1}  ~~+~~  \text{\small const} \cdot \tau^{n \,+\, 3 \lambda \,+\, 1}\,.
\end{align}

	The initial conditions for these equations are given by the values of the fields at some time $ \tau_* $ in the past
	(it should not matter whether it is the same $ \tau_* $ as before or not)
\begin{align}
%
\notag
	F^*_\lambda    & ~~=~~    \text{\small const}  ~~+~~  \text{\small const} \cdot \tau_*  ~~+~~  \text{\small const} \cdot \tau_*^{2 \lambda \,+\, 1} 
	\\[3.8mm]
%
	G^*_\lambda    & ~~=~~    \text{\small const}  ~~+~~  
				\text{\small const} \cdot \tau_*^{n \,+\, \lambda \,+\, 1}  ~~+~~  \text{\small const} \cdot \tau_*^{n \,+\, 3 \lambda \,+\, 1}\,.
\end{align}

	This determines the constants of integration in Eq.~\eqref{syssol},
\begin{align}
%
\label{sysfull}
	F_{\lambda}    & ~~=~~    F_\lambda^*  ~~+~~  \text{\small const} \cdot (\, \tau ~-~ \tau_* \,) 
						~~+~~  \text{\small const} \cdot (\, \tau^{n \,+\, \lambda \,+\, 1} ~-~ \tau_*^{n \,+\, \lambda \,+\, 1} \,)
	\\[3.8mm]
%
\notag
	G_{\lambda}    & ~~=~~    G_\lambda^*  ~~+~~  \text{\small const} \cdot (\, \tau^{n \,+\, \lambda \,+\, 1} ~-~ \tau_*^{n \,+\, \lambda \,+\, 1} \,)
						~~+~~  \dots\,.
\end{align}

	We presume that the magnetic field was generated {\it during} the inflation, most certainly, in the end.
	At time $ \tau_* $ therefore, the fields were negligible, 
\[
	F_\lambda^*    ~~\sim~~    G_\lambda^*    ~~\sim~~    0\,.
\]
	This allows us to re-write Eq.~\eqref{sysfull} as
\begin{align}
%
\notag
	F_{\lambda}(\tau)    & ~~=~~    \text{\small const} \cdot (\, \tau ~-~ \tau_* \,) 
					~~+~~  \text{\small const} \cdot (\, \tau^{n \,+\, \lambda \,+\, 1} ~-~ \tau_*^{n \,+\, \lambda \,+\, 1} \,)
	\\[3.8mm]
%
	G_{\lambda}(\tau)    & ~~=~~    \text{\small const} \cdot (\, \tau^{n \,+\, \lambda \,+\, 1} ~-~ \tau_*^{n \,+\, \lambda \,+\, 1} \,)
					~~+~~  \dots\,.
\end{align}
	
	Since $ \tau ~\propto~ e^{H t} $, we have an exponential approach to the values of the fields at the present time,
	which are
\begin{align}
%
\notag
	F_{\lambda}^{(0)}    & ~~\simeq~~    \text{\small const} \cdot \tau_*    ~~+~~    \text{\small const} \cdot \tau_*^{n \,+\, \lambda \,+\, 1} 
						~~+~~    \dots
	\\[3.8mm]
%
\label{solnow}
	G_{\lambda}^{(0)}    & ~~\simeq~~    \text{\small const} \cdot \tau_*^{n \,+\, \lambda \,+\, 1}    ~~+~~    \dots\,.
\end{align}

	Even though the inflation formally starts at $ \tau \,\to\, -\infty $, time $ \tau_* $ can be chosen to be finite, 
	such that the conditions \eqref{solnow} can be satisfied.

	We integrate the solution for $ F_\lambda $ over all the modes to obtain the magnetic field
\[
	\rho_B    ~~=~~    \frac{4\pi}{a^4(t_0)}\, \!\!
			{ \lower 2.8ex \hbox{    $   \stackrel{\displaystyle \stackrel{\;\;\;\;\;\; H a_0}{\int}}
							{\scriptstyle \!\!\!\!\!\! H a_i}  $    }    }
			\!\!\!\!\!\! \, dk\, k^4\, \big( F_{\lambda}^{(0)} \big)^2
			~~\simeq~~ 4\pi\, H^5\, a(t_0)\, \big( F_{\lambda}^{(0)} \big)^2\,.
\]

	We believe that the set of constants at our disposal --- $ n $, $ \lambda $, $ \tau_* $ and the proportionality constants
	are enough to provide a reasonable fit to the observed constraints on the magnetic field.
%	For this, generically, one can solve the system \eqref{diffeq} analytically, which is not too hard to do.
	The criterion of naturalness is that no constant is required to be on the order of $ 10^{60} $ or similar.

\end{document}
