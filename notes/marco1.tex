%\documentclass{article}
\documentclass[12pt]{article}
\usepackage{latexsym}
\usepackage{amsmath}
\usepackage{amssymb}
\usepackage{relsize}
\usepackage{geometry}
\geometry{letterpaper}

%\usepackage{showlabels}

\textwidth = 6.0 in
\textheight = 8.5 in
\oddsidemargin = 0.0 in
\evensidemargin = 0.0 in
\topmargin = 0.2 in
\headheight = 0.0 in
\headsep = 0.0 in
%\parskip = 0.05in
\parindent = 0.35in


%% common definitions
\def\stackunder#1#2{\mathrel{\mathop{#2}\limits_{#1}}}
\def\beqn{\begin{eqnarray}}
\def\eeqn{\end{eqnarray}}
\def\nn{\nonumber}
\def\baselinestretch{1.1}
\def\beq{\begin{equation}}
\def\eeq{\end{equation}}
\def\ba{\beq\new\begin{array}{c}}
\def\ea{\end{array}\eeq}
\def\be{\ba}
\def\ee{\ea}
\def\stackreb#1#2{\mathrel{\mathop{#2}\limits_{#1}}}
\def\Tr{{\rm Tr}}
\newcommand{\gsim}{\lower.7ex\hbox{$
\;\stackrel{\textstyle>}{\sim}\;$}}
\newcommand{\lsim}{\lower.7ex\hbox{$
\;\stackrel{\textstyle<}{\sim}\;$}}
\newcommand{\nfour}{${\mathcal N}=4$ }
\newcommand{\ntwo}{${\mathcal N}=2$ }
\newcommand{\ntwon}{${\mathcal N}=2$}
\newcommand{\ntwot}{${\mathcal N}= \left(2,2\right) $ }
\newcommand{\ntwoo}{${\mathcal N}= \left(0,2\right) $ }
\newcommand{\none}{${\mathcal N}=1$ }
\newcommand{\nonen}{${\mathcal N}=1$}
\newcommand{\vp}{\varphi}
\newcommand{\pt}{\partial}
\newcommand{\ve}{\varepsilon}
\newcommand{\gs}{g^{2}}
\newcommand{\qt}{\tilde q}

%%
\newcommand{\p}{\partial}
\newcommand{\wt}{\widetilde}
\newcommand{\ov}{\overline}
\newcommand{\mc}[1]{\mathcal{#1}}
\newcommand{\md}{\mathcal{D}}

\newcommand{\GeV}{{\rm GeV}}
\newcommand{\eV}{{\rm eV}}
\newcommand{\Heff}{{\mathcal{H}_{\rm eff}}}
\newcommand{\Leff}{{\mathcal{L}_{\rm eff}}}
\newcommand{\el}{{\rm EM}}
\newcommand{\uflavor}{\mathbf{1}_{\rm flavor}}
\newcommand{\lgr}{\left\lgroup}
\newcommand{\rgr}{\right\rgroup}

\newcommand{\Mpl}{M_{\rm Pl}}
\newcommand{\suc}{{{\rm SU}_{\rm C}(3)}}
\newcommand{\sul}{{{\rm SU}_{\rm L}(2)}}
\newcommand{\sutw}{{\rm SU}(2)}
\newcommand{\suth}{{\rm SU}(3)}
\newcommand{\ue}{{\rm U}(1)}
%%%%%%%%%%%%%%%%%%%%%%%%%%%%%%%%%%%%%%%
%  Slash character...
\def\slashed#1{\setbox0=\hbox{$#1$}             % set a box for #1
   \dimen0=\wd0                                 % and get its size
   \setbox1=\hbox{/} \dimen1=\wd1               % get size of /
   \ifdim\dimen0>\dimen1                        % #1 is bigger
      \rlap{\hbox to \dimen0{\hfil/\hfil}}      % so center / in box
      #1                                        % and print #1
   \else                                        % / is bigger
      \rlap{\hbox to \dimen1{\hfil$#1$\hfil}}   % so center #1
      /                                         % and print /
   \fi}                                        %

%%EXAMPLE:  $\slashed{E}$ or $\slashed{E}_{t}$

%%

\newcommand{\LN}{\Lambda_\text{SU($N$)}}
\newcommand{\sunu}{{\rm SU($N$) $\times$ U(1)} }
\newcommand{\sunun}{{\rm SU($N$) $\times$ U(1)}}
\def\cfl {$\text{SU($N$)}_{\rm C+F}$ }
\newcommand{\mUp}{m_{\rm U(1)}^{+}}
\newcommand{\mUm}{m_{\rm U(1)}^{-}}
\newcommand{\mNp}{m_\text{SU($N$)}^{+}}
\newcommand{\mNm}{m_\text{SU($N$)}^{-}}
\newcommand{\AU}{\mc{A}^{\rm U(1)}}
\newcommand{\AN}{\mc{A}^\text{SU($N$)}}
\newcommand{\aU}{a^{\rm U(1)}}
\newcommand{\aN}{a^\text{SU($N$)}}
\newcommand{\baU}{\ov{a}{}^{\rm U(1)}}
\newcommand{\baN}{\ov{a}{}^\text{SU($N$)}}
\newcommand{\lU}{\lambda^{\rm U(1)}}
\newcommand{\lN}{\lambda^\text{SU($N$)}}
%\newcommand{\Tr}{{\rm Tr\,}}
\newcommand{\bxir}{\ov{\xi}{}_R}
\newcommand{\bxil}{\ov{\xi}{}_L}
\newcommand{\xir}{\xi_R}
\newcommand{\xil}{\xi_L}
\newcommand{\bzl}{\ov{\zeta}{}_L}
\newcommand{\bzr}{\ov{\zeta}{}_R}
\newcommand{\zr}{\zeta_R}
\newcommand{\zl}{\zeta_L}
\newcommand{\nbar}{\ov{n}}

\newcommand{\loU}{\lambda_0^{\rm U(1)}}
\newcommand{\llU}{\lambda_1^{\rm U(1)}}
\newcommand{\loN}{\lambda_0^\text{SU($N$)}}
\newcommand{\llN}{\lambda_1^\text{SU($N$)}}
\newcommand{\poU}{\psi_0^{\rm U(1)}}
\newcommand{\plU}{\psi_1^{\rm U(1)}}
\newcommand{\poN}{\psi_0^\text{SU($N$)}}
\newcommand{\plN}{\psi_1^\text{SU($N$)}}

\newcommand{\CPC}{CP($N-1$)$\times$C }
\newcommand{\CPCn}{CP($N-1$)$\times$C}

\newcommand{\MN}{M_\text{SU($N$)}}
\newcommand{\MU}{M_{\rm U(1)}}

\newcommand{\tgamma}{\wt{\gamma}}
\newcommand{\btgamma}{\ov{\tgamma}}

\begin{document}

	We accept that the electric and magnetic fields are
\[
	E_k  ~~~~=~~~~  -\, \frac{I}{a^2}\, \bigg( \frac V I \bigg)^\prime\,,
	\qquad\qquad
	B_k  ~~~~=~~~~  \pm\, \frac{k}{a^2}\, V
\]
	and their contributions to the energy density are
\begin{align*}
%
	\rho_E  &  ~~~~=~~~~  \frac{4\pi}{a^4}\, I^2\, \int dk\, k^2\,  \bigg| \left(\frac V I \right)^\prime \bigg|^2
	\\
%
	\rho_B  &  ~~~~=~~~~  \frac{4\pi}{a^4}\, I^2\, \int dk\, k^4\,  \bigg|\; \frac V I \;\bigg|^2
\end{align*}
	These are the formulas from your email.

	Now let us suppose, as before, that
\[
	I  ~~~~\propto~~~~  \tau^n\,,
	\qquad
	\text{and}
	\qquad
	V_\lambda  ~~~~\propto~~~~  \tau^m\\,
\]
	where $ n $ and $ m $ are integers and of course $ m $ will depend on $ n $, being a solution of the equations of motion.
	They can be positive or negative, does not matter now.

	Now,
\[
	V/I  ~~~~=~~~~  \tau^{m-n}
	\qquad
	\text{hence}
	\qquad
	(V/I)^\prime  ~~~~=~~~~  \tau^{m-n-1}\,,
\]
	up to numerical coefficients.

	The densities are, multiplied by the same (functional) proportionality coefficients,
\begin{align*}
%
	\rho_E  & ~~~~\propto~~~~ \int dk\, k^2\, \big(\tau^{m-n-1}\big)^2  ~~~~=~~~~ \int_{Ha_i}^{Ha_0}\, dk\, k^2 \cdot \tau^{2m - 2n - 2} \\
%
	\rho_B  & ~~~~\propto~~~~ \int dk\, k^4\, \big(\tau^{m-n}\big)^2    ~~~~=~~~~ \int_{Ha_i}^{Ha_0}\, dk\, k^4 \cdot \tau^{2m - 2n}
\end{align*}

	And hence,
\begin{align*}
%
	\rho_E  & ~~~~\propto~~~~ \big(Ha_0\big)^3 \cdot \tau^{2m - 2n - 2}  ~~~~\propto~~~~  \tau^{2m-2n-5} \\
%
	\rho_B  & ~~~~\propto~~~~ \big(Ha_0\big)^5 \cdot \tau^{2m - 2n}  ~~~~\propto~~~~ \tau^{2m-2n-5}
\end{align*}
	I don't see how one is growing much bigger than the other.

\pagebreak
	Mukhanov did not write the following. 
	In the {\it strong coupling} case, he never mentioned that electric field becomes much bigger than the magnetic field.
	He did say that the coupling becomes strong --- the known issue --- but pretty much nothing else 
	(now in the case of weak coupling --- which I am not discussing --- he did say that there is an excess of electromagnetic energy
	compared to the inflationary potential). 
	
\vspace{8mm}
	Imagine a regular electromagnetism where we do not have the restriction that $ e_0 $ is order one today, but can be 
	accepted to be negligible.
	Let it be order one before the inflation and zero after the inflation 
	(the same ``strong coupling'' scenario but now $ e $  is renormalized so it is weak at all times).
	Then there does not seem to be a problem that the electric field greatly exceeds the magnetic field, 
	nor that their overall density disturbs the inflation.
	Do you agree?

\end{document}
