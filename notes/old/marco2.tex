%\documentclass{article}
\documentclass[12pt]{article}
\usepackage{latexsym}
\usepackage{amsmath}
\usepackage{amssymb}
\usepackage{relsize}
\usepackage{geometry}
\usepackage{picins}
\geometry{letterpaper}

%\usepackage{showlabels}

\textwidth = 6.0 in
\textheight = 8.5 in
\oddsidemargin = 0.0 in
\evensidemargin = 0.0 in
\topmargin = 0.2 in
\headheight = 0.0 in
\headsep = 0.0 in
%\parskip = 0.05in
\parindent = 0.35in


%% common definitions
\def\stackunder#1#2{\mathrel{\mathop{#2}\limits_{#1}}}
\def\beqn{\begin{eqnarray}}
\def\eeqn{\end{eqnarray}}
\def\nn{\nonumber}
\def\baselinestretch{1.1}
\def\beq{\begin{equation}}
\def\eeq{\end{equation}}
\def\ba{\beq\new\begin{array}{c}}
\def\ea{\end{array}\eeq}
\def\be{\ba}
\def\ee{\ea}
\def\stackreb#1#2{\mathrel{\mathop{#2}\limits_{#1}}}
\def\Tr{{\rm Tr}}
\newcommand{\gsim}{\lower.7ex\hbox{$
\;\stackrel{\textstyle>}{\sim}\;$}}
\newcommand{\lsim}{\lower.7ex\hbox{$
\;\stackrel{\textstyle<}{\sim}\;$}}

%%
\newcommand{\p}{\partial}
\newcommand{\wt}{\widetilde}
\newcommand{\ov}{\overline}
\newcommand{\mc}[1]{\mathcal{#1}}
\newcommand{\md}{\mathcal{D}}
\newcommand{\ph}{\phantom}

\newcommand{\GeV}{{\rm GeV}}
\newcommand{\eV}{{\rm eV}}
\newcommand{\Heff}{{\mathcal{H}_{\rm eff}}}
\newcommand{\Leff}{{\mathcal{L}_{\rm eff}}}
\newcommand{\el}{{\rm EM}}
\newcommand{\uflavor}{\mathbf{1}_{\rm flavor}}
\newcommand{\lgr}{\left\lgroup}
\newcommand{\rgr}{\right\rgroup}

\newcommand{\Mpl}{M_{\rm Pl}}
\newcommand{\suc}{{{\rm SU}_{\rm C}(3)}}
\newcommand{\sul}{{{\rm SU}_{\rm L}(2)}}
\newcommand{\sutw}{{\rm SU}(2)}
\newcommand{\suth}{{\rm SU}(3)}
\newcommand{\ue}{{\rm U}(1)}
%%%%%%%%%%%%%%%%%%%%%%%%%%%%%%%%%%%%%%%
%  Slash character...
\def\slashed#1{\setbox0=\hbox{$#1$}             % set a box for #1
   \dimen0=\wd0                                 % and get its size
   \setbox1=\hbox{/} \dimen1=\wd1               % get size of /
   \ifdim\dimen0>\dimen1                        % #1 is bigger
      \rlap{\hbox to \dimen0{\hfil/\hfil}}      % so center / in box
      #1                                        % and print #1
   \else                                        % / is bigger
      \rlap{\hbox to \dimen1{\hfil$#1$\hfil}}   % so center #1
      /                                         % and print /
   \fi}                                        %

%%EXAMPLE:  $\slashed{E}$ or $\slashed{E}_{t}$

%%

\newcommand{\LN}{\Lambda_\text{SU($N$)}}
\newcommand{\sunu}{{\rm SU($N$) $\times$ U(1)} }
\newcommand{\sunun}{{\rm SU($N$) $\times$ U(1)}}
\def\cfl {$\text{SU($N$)}_{\rm C+F}$ }
\newcommand{\mUp}{m_{\rm U(1)}^{+}}
\newcommand{\mUm}{m_{\rm U(1)}^{-}}
\newcommand{\mNp}{m_\text{SU($N$)}^{+}}
\newcommand{\mNm}{m_\text{SU($N$)}^{-}}
\newcommand{\AU}{\mc{A}^{\rm U(1)}}
\newcommand{\AN}{\mc{A}^\text{SU($N$)}}
\newcommand{\aU}{a^{\rm U(1)}}
\newcommand{\aN}{a^\text{SU($N$)}}
\newcommand{\baU}{\ov{a}{}^{\rm U(1)}}
\newcommand{\baN}{\ov{a}{}^\text{SU($N$)}}
\newcommand{\lU}{\lambda^{\rm U(1)}}
\newcommand{\lN}{\lambda^\text{SU($N$)}}
%\newcommand{\Tr}{{\rm Tr\,}}
\newcommand{\bxir}{\ov{\xi}{}_R}
\newcommand{\bxil}{\ov{\xi}{}_L}
\newcommand{\xir}{\xi_R}
\newcommand{\xil}{\xi_L}
\newcommand{\bzl}{\ov{\zeta}{}_L}
\newcommand{\bzr}{\ov{\zeta}{}_R}
\newcommand{\zr}{\zeta_R}
\newcommand{\zl}{\zeta_L}
\newcommand{\nbar}{\ov{n}}

\newcommand{\loU}{\lambda_0^{\rm U(1)}}
\newcommand{\llU}{\lambda_1^{\rm U(1)}}
\newcommand{\loN}{\lambda_0^\text{SU($N$)}}
\newcommand{\llN}{\lambda_1^\text{SU($N$)}}
\newcommand{\poU}{\psi_0^{\rm U(1)}}
\newcommand{\plU}{\psi_1^{\rm U(1)}}
\newcommand{\poN}{\psi_0^\text{SU($N$)}}
\newcommand{\plN}{\psi_1^\text{SU($N$)}}

\newcommand{\CPC}{CP($N-1$)$\times$C }
\newcommand{\CPCn}{CP($N-1$)$\times$C}

\newcommand{\MN}{M_\text{SU($N$)}}
\newcommand{\MU}{M_{\rm U(1)}}

\newcommand{\tgamma}{\wt{\gamma}}
\newcommand{\btgamma}{\ov{\tgamma}}

\begin{document}

\centerline{\Huge\bf The Central Idea}

\vspace{2.0mm}

\begin{minipage}[b]{0.76cm}
                \fontseries{bx}
                \fontfamily{pnc}
                \fontsize{44}{12}
                \selectfont
                T
\end{minipage}
\begin{minipage}[b]{3.36cm}
        	\Large\fontfamily{pnc}\selectfont
		\textsc{he kinetic}
\end{minipage}
	mixing matrix can be specified in an explicitly diagonalized form
\beq
	\mc{M}  ~~~~=~~~~  
			\lgr\; \begin{matrix}
				\cos\, \theta    &    - \sin\, \theta \\
				\sin\, \theta    &    \ph- \cos\, \theta
			\end{matrix} \;\rgr
			\lgr\; \begin{matrix}
			     		a^2    &        \\
					       &    b^2
			\end{matrix} \;\rgr
			\lgr\; \begin{matrix}
				\ph-\cos\, \theta    &    \sin\, \theta \\
				- \sin\, \theta    &    \cos\, \theta
			\end{matrix} \;\rgr
			.
\eeq

	The quantity $ a^2 $ will be the effective coupling constant of the photon, and therefore
	we choose 
\[
	a^2  ~~\simeq~~  1\,.
\]
	Meanwhile, the quantity $ b^2 $ will be the coupling constant of the hidden photon.
	We would like to have the hidden photon completely decoupled by the end of inflation, and also
	never to be at the strong coupling.
	Therefore, we choose $ b^2 $ to be growing,
\[
	b^2  ~~\propto~~ a^n(\tau)\,.
\]

	We can assume that the angle $ \theta $ is small,
\[
	\theta  ~~\ll~~  1\,,
\]
	although I do not believe this is necessary.

	We obtain the convenient form of the kinetic mixing matrix
\beq
	\mc{M}  ~~=~~  
			\lgr\; \begin{matrix}
				    g^2         &    \epsilon    \\
				    \epsilon    &    f^2
			     \end{matrix} \;\rgr,
\eeq
	where
\[
	g^2  ~~\ll~~  f^2
	\qquad\qquad
	\text{and}
	\qquad\qquad
	\epsilon  ~~\ll~~  f^2\,.
\]


\vspace{1.0cm}
\centerline{*\qquad\qquad\qquad*\qquad\qquad\qquad*}
\vspace{1.0cm}

	
	The kinetic term is then
\beq
	\lgr F_{\mu\nu}\, \quad G_{\mu\nu} \rgr 
	\lgr\; \begin{matrix}
				g^2        &    \epsilon  \\
				\epsilon   &    f^2
	\end{matrix} \;\rgr
	\lgr\, \begin{matrix}
		F_{\mu\nu}  \\
		G_{\mu\nu}
	     \end{matrix} \,\rgr,
\eeq
	where $ F_{\mu\nu} $ and $ G_{\mu\nu} $ are the original mixed fields.
	After diagonalization this kinetic term leads to
\beq
	a^2\, A_{\mu\nu}^2  ~~+~~  b^2\, B_{\mu\nu}^2  ~~+~~  \text{one-derivative term.}
\eeq
%	The first term would then be the kinetic term for the photon, 
%	while the (exponentially decreasing) second term is interpreted as the kinetic term
%	for the hidden photon.


\vspace{1.0cm}
\centerline{*\qquad\qquad\qquad*\qquad\qquad\qquad*}
\vspace{1.0cm}

	
	Let us call the undiagonalized fields $ \Phi_\mu $,
\beq
	\lgr  F_\mu\,, \quad G_\mu  \rgr    ~~\equiv~~    \Phi_\mu\,.
\eeq


	As we know, at the time when a mode has exited the horizon, the equation of motion for
	it can be written as
\beq
	\big(\, \mc{M} \cdot \Phi^\prime \,\big)^\prime    ~~=~~    0\,,
	\qquad\qquad
	\text{or}
	\qquad\qquad
	\mc{M} \cdot \Phi^\prime    ~~=~~    \text{const}.
\eeq

	That is, fixing some time $ \tau_* $, we have
\beq
	\mc{M} \cdot \Phi^\prime    ~~=~~    \mc{M}_* \cdot \Phi_*
	\qquad\qquad
	\text{for that $\tau_*$}.
\eeq

	This leads to the differential equation
\beq
	\Phi^\prime  ~~=~~ \mc{M}^{-1} \cdot \mc{M}_*\, \Phi_*^\prime\,.
\eeq

	The right hand side of this equation is in principle a known function of time, as long as
	we find the inverse of matrix $ \mc{M} $.
	The rest of the right hand side is just a set of constants.
	We expect the inverse of $ \mc{M} $ to be expressible just as a combination of powers
	of $ a(\tau) $ --- that is, a combination of powers of $ \tau $ itself.


\vspace{1.0cm}
\centerline{*\qquad\qquad\qquad*\qquad\qquad\qquad*}
\vspace{1.0cm}

	
	Quite generally,
\beq
	\mc{M}^{-1}   ~~=~~    \frac{1}{g^2 f^2 \,-\, \epsilon^2}\,
					\lgr \begin{matrix}
						\ph{-} f^2    &    -\epsilon    \\
						-\epsilon   &    \ph- g^2
					\end{matrix} \;\rgr  .
\eeq
	
	The denominator can just be expanded,
\beq
	\frac{1}{g^2 f^2 \,-\, \epsilon^2}    ~~=~~    
		\frac{1}{g^2 f^2}\, 
		\lgr  1  ~~+~~  \frac{\epsilon^2}{g^2f^2}  ~~+~~  \frac{\epsilon^4}{g^4 f^4}  ~~+~~  \dots \rgr,
\eeq
	and only the leading term need to be retained.

	Of course this leading term, as well as the rest of the matrix $ \mc{M}^{-1} $ is just expressed
	as a set of powers of $ \tau $.

\end{document}
