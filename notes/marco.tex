%\documentclass{article}
\documentclass[12pt]{article}
\usepackage{latexsym}
\usepackage{amsmath}
\usepackage{amssymb}
\usepackage{relsize}
\usepackage{geometry}
\geometry{letterpaper}

%\usepackage{showlabels}

\textwidth = 6.0 in
\textheight = 8.5 in
\oddsidemargin = 0.0 in
\evensidemargin = 0.0 in
\topmargin = 0.2 in
\headheight = 0.0 in
\headsep = 0.0 in
%\parskip = 0.05in
\parindent = 0.35in


%% common definitions
\def\stackunder#1#2{\mathrel{\mathop{#2}\limits_{#1}}}
\def\beqn{\begin{eqnarray}}
\def\eeqn{\end{eqnarray}}
\def\nn{\nonumber}
\def\baselinestretch{1.1}
\def\beq{\begin{equation}}
\def\eeq{\end{equation}}
\def\ba{\beq\new\begin{array}{c}}
\def\ea{\end{array}\eeq}
\def\be{\ba}
\def\ee{\ea}
\def\stackreb#1#2{\mathrel{\mathop{#2}\limits_{#1}}}
\def\Tr{{\rm Tr}}
\newcommand{\gsim}{\lower.7ex\hbox{$
\;\stackrel{\textstyle>}{\sim}\;$}}
\newcommand{\lsim}{\lower.7ex\hbox{$
\;\stackrel{\textstyle<}{\sim}\;$}}
\newcommand{\nfour}{${\mathcal N}=4$ }
\newcommand{\ntwo}{${\mathcal N}=2$ }
\newcommand{\ntwon}{${\mathcal N}=2$}
\newcommand{\ntwot}{${\mathcal N}= \left(2,2\right) $ }
\newcommand{\ntwoo}{${\mathcal N}= \left(0,2\right) $ }
\newcommand{\none}{${\mathcal N}=1$ }
\newcommand{\nonen}{${\mathcal N}=1$}
\newcommand{\vp}{\varphi}
\newcommand{\pt}{\partial}
\newcommand{\ve}{\varepsilon}
\newcommand{\gs}{g^{2}}
\newcommand{\qt}{\tilde q}

%%
\newcommand{\p}{\partial}
\newcommand{\wt}{\widetilde}
\newcommand{\ov}{\overline}
\newcommand{\mc}[1]{\mathcal{#1}}
\newcommand{\md}{\mathcal{D}}

\newcommand{\GeV}{{\rm GeV}}
\newcommand{\eV}{{\rm eV}}
\newcommand{\Heff}{{\mathcal{H}_{\rm eff}}}
\newcommand{\Leff}{{\mathcal{L}_{\rm eff}}}
\newcommand{\el}{{\rm EM}}
\newcommand{\uflavor}{\mathbf{1}_{\rm flavor}}
\newcommand{\lgr}{\left\lgroup}
\newcommand{\rgr}{\right\rgroup}

\newcommand{\Mpl}{M_{\rm Pl}}
\newcommand{\suc}{{{\rm SU}_{\rm C}(3)}}
\newcommand{\sul}{{{\rm SU}_{\rm L}(2)}}
\newcommand{\sutw}{{\rm SU}(2)}
\newcommand{\suth}{{\rm SU}(3)}
\newcommand{\ue}{{\rm U}(1)}
%%%%%%%%%%%%%%%%%%%%%%%%%%%%%%%%%%%%%%%
%  Slash character...
\def\slashed#1{\setbox0=\hbox{$#1$}             % set a box for #1
   \dimen0=\wd0                                 % and get its size
   \setbox1=\hbox{/} \dimen1=\wd1               % get size of /
   \ifdim\dimen0>\dimen1                        % #1 is bigger
      \rlap{\hbox to \dimen0{\hfil/\hfil}}      % so center / in box
      #1                                        % and print #1
   \else                                        % / is bigger
      \rlap{\hbox to \dimen1{\hfil$#1$\hfil}}   % so center #1
      /                                         % and print /
   \fi}                                        %

%%EXAMPLE:  $\slashed{E}$ or $\slashed{E}_{t}$

%%

\newcommand{\LN}{\Lambda_\text{SU($N$)}}
\newcommand{\sunu}{{\rm SU($N$) $\times$ U(1)} }
\newcommand{\sunun}{{\rm SU($N$) $\times$ U(1)}}
\def\cfl {$\text{SU($N$)}_{\rm C+F}$ }
\newcommand{\mUp}{m_{\rm U(1)}^{+}}
\newcommand{\mUm}{m_{\rm U(1)}^{-}}
\newcommand{\mNp}{m_\text{SU($N$)}^{+}}
\newcommand{\mNm}{m_\text{SU($N$)}^{-}}
\newcommand{\AU}{\mc{A}^{\rm U(1)}}
\newcommand{\AN}{\mc{A}^\text{SU($N$)}}
\newcommand{\aU}{a^{\rm U(1)}}
\newcommand{\aN}{a^\text{SU($N$)}}
\newcommand{\baU}{\ov{a}{}^{\rm U(1)}}
\newcommand{\baN}{\ov{a}{}^\text{SU($N$)}}
\newcommand{\lU}{\lambda^{\rm U(1)}}
\newcommand{\lN}{\lambda^\text{SU($N$)}}
%\newcommand{\Tr}{{\rm Tr\,}}
\newcommand{\bxir}{\ov{\xi}{}_R}
\newcommand{\bxil}{\ov{\xi}{}_L}
\newcommand{\xir}{\xi_R}
\newcommand{\xil}{\xi_L}
\newcommand{\bzl}{\ov{\zeta}{}_L}
\newcommand{\bzr}{\ov{\zeta}{}_R}
\newcommand{\zr}{\zeta_R}
\newcommand{\zl}{\zeta_L}
\newcommand{\nbar}{\ov{n}}

\newcommand{\loU}{\lambda_0^{\rm U(1)}}
\newcommand{\llU}{\lambda_1^{\rm U(1)}}
\newcommand{\loN}{\lambda_0^\text{SU($N$)}}
\newcommand{\llN}{\lambda_1^\text{SU($N$)}}
\newcommand{\poU}{\psi_0^{\rm U(1)}}
\newcommand{\plU}{\psi_1^{\rm U(1)}}
\newcommand{\poN}{\psi_0^\text{SU($N$)}}
\newcommand{\plN}{\psi_1^\text{SU($N$)}}

\newcommand{\CPC}{CP($N-1$)$\times$C }
\newcommand{\CPCn}{CP($N-1$)$\times$C}

\newcommand{\MN}{M_\text{SU($N$)}}
\newcommand{\MU}{M_{\rm U(1)}}

\newcommand{\tgamma}{\wt{\gamma}}
\newcommand{\btgamma}{\ov{\tgamma}}

\begin{document}

	Consider just the electromagnetic potential of {\it one} of the U(1) fields, say $ A_\mu $.
	Then the long-wave solution will look like
\beq
	A_\lambda    ~~\propto~~    \tau^n\,,  \qquad\qquad n ~\neq~ 0\,
\eeq	
	with some $ n $.
	Correspondingly,
\beq
	A_\lambda^{~\prime}  ~~\propto~~  \tau^{n-1}.
\eeq

	We have then,
\begin{align}
%
	\rho_E  &  ~~\propto~~  \frac{\langle\, 0\, |\, \vec{E}^2\, |\, 0\, \rangle}{ 2\, a^4  }  ~~\propto~~  \int\, k^2\, dk\,  \tau^{2n-2} \,,
	\\
%
	\rho_B  &  ~~\propto~~  \frac{\langle\, 0\, |\, \vec{B}^2\, |\, 0\, \rangle}{ 2\, a^4 }  ~~\propto~~  \int\, k^4\, dk\,  \tau^{2n}\,,
\end{align}
	where $ k^2\, dk $ is basically the remnant of the Fourier integrals, and
	the extra factor of $ k^2 $ in $ \rho_B $ comes from the spatial derivative in $ \vec{B}  ~=~  [\vec{\nabla} \vec{A}] $.

	The integral over $ k $ goes over all the inflation,
\begin{align}
%
	\rho_E  &  ~~\propto~~  \int_{H a_i}^{H a_0}\, k^2\, dk\,  \tau^{2n-2}  
		~~\propto~~  (H  a_0)^3 \cdot \tau_0^{2n-2}  ~~=~~  \tau_0^{2n-5}\,,  \\
%
	\rho_B  &  ~~\propto~~  \int_{H a_i}^{H a_0}\, k^4\, dk\,  \tau^{2n}
		~~\propto~~  (H  a_0)^5 \cdot \tau_0^{2n}  ~~=~~  \tau_0^{2n-5}\,.
\end{align}
	In this case we have that $ \rho_E ~\to~ 0 $ at the same rate as $ \rho_B $

%	Mukhanov and Rubinstein did run into the strong coupling problem, but they never mentioned the simultaneous problem with the overproduction
%	of the electric field. 
%
%	I can't help thinking that there cannot be a way of cooking up such a mixture of the two U(1) fields and such a power of $ n $,
%	{\it and} such a coupling to the matter, that there is a magnetic field generated and yet only one of the fields couples to the 
%	matter at the present time. 
%	The coupling to the matter in the very original Lagrangian is not fixed and can be rotated/defined in an arbitrary way, in particular,
%	such that at the present time it is just one field which couples to the matter

\end{document}
